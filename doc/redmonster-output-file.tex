\documentclass[12pt]{article}
\parindent=0pt
\parskip=8pt
\begin{document}

\title{Documentation of \texttt{redmonster} Output Files}

\author{Timothy A. Hutchinson}

\maketitle

\section{Introduction}

This document describes the format and content of the files
generated by the \texttt{redmonster} software.

The principal output of the software is the \textit{redmonster} file.  It contains all of the
spectroscopic redshift and classification information, as well as the models used for the fits.
This file is described in $\S$2.

By default, the software does not store the $\chi^2$ surfaces generated by
the fitting routines.  However, the software does provide the option to write these surfaces
to a separate \textit{chi2arr} file.  This file is described in $\S$3.

\section{\textit{redmonster} File}

\subsection{Naming convention}

Primary output files from \texttt{redmonster} will generally follow the naming scheme

%\hspace*{36pt}\textit{redmonster-pppp-mmmmm.fits}
\begin{center}
	\textit{redmonster-pppp-mmmmm.fits}
\end{center}

where \textit{pppp} is the 4-digit SDSS plate number, and \textit{mmmmm}
is the 5-digit MJD.  These correspond with the \textit{spPlate} input file and
and cannot be changed by the user.

When a file with the above naming convention exists in the specified directory, 
the default behavior of the software is to overwrite it.  However, the user may 
elect to leave the older file intact, in which case the new file will be written as

%\hspace*{36pt}\textit{redmonster-pppp-mmmmm-YYYY-nn-dd\_HH:MM:SS.fits}
\begin{center}
	\textit{redmonster-pppp-mmmmm-YYYY-nn-dd\_HH:MM:SS.fits}
\end{center}

where \textit{YYYY}, \textit{nn}, \textit{dd}, \textit{HH}, \textit{MM}, and \textit{SS}
are the year, month, day, hour, minute, and second, respectively, of the time (GMT) at which
the file was written.

\subsection{File type}

All \texttt{redmonster} outputs are uncompressed FITS files with all relevant information
in the primary HDU header, a BIN table, and an imageHDU.  The file size is approximately 20 MB.

\subsection{File structure}

The general structure of the file is as follows:

\begin{center}
	\begin{tabular}{ | l | l | l |}
	\hline
	HDU0 & NULL & Empty \\ \hline
	HDU1 & Binary FITS Table & Object redshifts and classifications \\ \hline
	HDU2 & \texttt{nfibers} $\times$ \texttt{npix} float image & Best-fit template model for each object \\
	\hline
	\end{tabular}
\end{center}

\subsection{Header}

The primary HDU header is nearly identical to that of the \textit{spPlate}
files\footnote{Documented at http://data.sdss3.org/datamodel/index-files.html}, with a few minor
additions specific to \texttt{redmonster}.  Those additions are as follows:

\begin{center}
	\begin{tabular}{ | l | l | }
	\hline
	VERS\_RM & Version of \texttt{redmonster} that produced the file \\ \hline
	DATE\_RM & Date and time (GMT) of \texttt{redmonster} completion \\ \hline
	NFIBERS & Number of fibers used \\
	\hline
	\end{tabular}
\end{center}

\subsection{Binary table}

The binary FITS table contains all redshift and classification information.
The data included is as follows, specified by the binary table's header:

\begin{center}
	\begin{tabular}{ | l | l | }
	\hline
	Z1 & Best redshift (least $\chi_r^2$) \\ \hline
	Z2 & Second best redshift \\ \hline
	Z\_ERR1 & 1-$\sigma$ error associated with Z1 \\ \hline
	Z\_ERR2 & 1-$\sigma$ error associated with Z2 \\ \hline
	CLASS & Object type classification \\ \hline
	SUBCLASS & Best-fit template parameters \\ \hline
	FIBERID & \textit{spPlate} fiber numbers used (0-based) \\ \hline
	MINVECTOR & Coordinates of best-fit template in \textit{ndArch} file \\ \hline
	ZWARNING & Warning flags (identical to BOSS flags) \\ \hline
	DOF & Degrees of freedom used in calculating $\chi_r^2$ \\ \hline
	NPOLY & Number of additive polynomials used \\ \hline
	FNAME & Full name of \textit{ndArch} file of best-fit template \\ \hline
	NPIXSTEP & Pixel step size in redshift space used \\
	\hline
	\end{tabular}
\end{center}

There are a few things to note about the values in the BIN table.  First, in some cases
(failure of fit, small $\Delta\chi^2$, etc.), redshift values have been set to -1.  In
these cases, the associated errors have also been set to -1.  Second, for \textit{ndArch}
files with multiple parameter dimensions, the dictionaries corresponding to those dimensions
have been converted into a string when written to SUBCLASS.  The tuples comprising the
values of MINVECTOR have been treated the same way.  Using the built-in python
function \texttt{eval()} will convert them back to dictionaries and tuples, respectively.

\subsection{ImageHDU}

The imageHDU contains an \texttt{nfibers} $\times$ \texttt{npix} float
image comprised of the best-fit models (template + polynomials at global
minimum of $\chi^2$ surface) for each fiber.  The models are in units of
$10^{-17} erg\ cm^{-2} s^{-1} \AA^{-1}$

\section{\textit{chi2arr} File}

\subsection{Naming convention}

The \textit{chi2arr} file follows the naming scheme

%\hspace*{36pt}\textit{chi2arr-tttttttttttt-pppp-mmmmm.fits}
\begin{center}
	\textit{chi2arr-tttttttttttt-pppp-mmmmm.fits}
\end{center}

where \textit{tttttttttttt} is the template's type (derived from the \textit{ndArch} file),
\textit{pppp} is the 4-digit SDSS plate number, and \textit{mmmmm} is the 5-digit
MJD.  As in $\S$2.1, plate number and MJD are taken from the input \textit{spPlate} file.
Currently, there is no secondary naming scheme, and older \textit{chi2arr} files
will be overwritten.

\subsection{File Type}

All \texttt{chi2arr} files are in the uncompressed FITS format.  All data is located in the
primary HDU.

The file size varies depending on number of template parameters and redshift range,
but can quickly become quite large.  For example, fitting 300 templates to 595 fibers at
3469 redshifts results in a 4.7 GB file.

\subsection{File structure}

The primary HDU contains a \texttt{numpy} array of $\chi^2$ values of dimensions \\
\texttt{nfibers} $\times$
\texttt{npar$_1$} $\times$ ... $\times$ \texttt{npar$_N$} $\times$ \texttt{n$_z$} ,\\
where \texttt{npar$_i$} is the length of the $i^{\mathrm{th}}$ parameter dimension
and \texttt{n$_z$} is the number of redshifts explored.

There is no information in the header.

\end{document}











