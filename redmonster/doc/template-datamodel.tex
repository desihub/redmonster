\documentclass[12pt]{article}
\parindent=0pt
\parskip=8pt
\begin{document}

\title{Proposed data model for a general archetype
redshift template file format}

\author{Adam S. Bolton}

\maketitle

\section{Introduction}

This note describes a proposed file format for storing general
``archetype'' redshift template grids for ingestion into
the \texttt{redmonster} classification, redshift, and parameter
measurement Python code.

The proposed standard is motivated by the desire to provide a general
form of template file that can be written by end users and ingested
into the redshift software to allow highly configurable spectroscopic
template classes without any recoding of the low-level analysis routines.

``Archetypes'' refers to sets of single spectral template vectors that are not
fit in linear combination with any other templates (except possibly
low-order nuisance vectors).

This file standard is oriented towards the familiar units and conventions
of optical spectroscopic redshift measurement, which may not be convenient
for other experimental regimes.

\section{File Format}

\subsection{File naming convention}

Files conforming to this data model will follow the naming scheme

\hspace*{36pt}\texttt{ndArch-CLASS-VERSION.fits}

The string ``\texttt{ndArch}'' signifies ``N-dimensional archetype'',
reflecting the accommodation of multiple physical parameter dimensions
in the file standard.

\texttt{CLASS} and \texttt{VERSION} are arbitrary
strings that may contain any legal filename
characters other than the dash (\texttt{-}).  Both the \texttt{CLASS}
and \texttt{VERSION} strings are required to be present in
some non-trivial form.  The \texttt{CLASS} string will be extracted
and used by the \texttt{redmonster} code as the primary qualitative
identifier of the class of models contained in the file,
while the \texttt{VERSION} string is intended to distinguish
between multiple versions of the same type of model file.
Users may adopt further specific conventions for \texttt{CLASS}
and \texttt{VERSION} as appropriate to their needs.

The exact extension string ``\texttt{.fits}'' is formally required,
but may in the future be relaxed to permit alternate and/or
compressed versions.

\subsection{File type}

Valid \texttt{ndArch} files will be uncompressed FITS
files with all relevant information contained in the
filename and the primary HDU\@.

Additional FITS HDU extensions may be present, but are not defined
by this data model.

\subsection{Data structure and requirements}

The data contained in the \texttt{ndArch} file
will consist of a single multi-dimensional array
containing flux densities or luminosity densities in units of
$F_{\lambda}$ (power per unit area per unit wavelength) or $L_{\lambda}$
(power per unit wavelength).
The absolute normalization may be physically
meaningful, but is not required to be meaningful.
If the normalization is meaningful, the units
shall be specified via the \texttt{BUNIT} keyword (see below),
and must be the same for all data values in the file.

The first axis of the data array shall correspond to \textbf{vacuum}
wavelength, and shall be gridded in positive increments of constant
log-wavelength.  There may be zero or more axes in addition
to the first axis, up to the maximum number allowed by
the FITS standard.  Each axis beyond the first will generally correspond
to a monotonically ordered
physical model-parameter dimension (age, metallicity,
emission-line strength, etc.), but may also correspond to
an arbitrary labeled or unlabeled collection.
Conventions for specifying the parameter dimensions are
described in the following section.

The archetype template vectors shall be assumed to have
a uniform resolution characterized by a Gaussian
line-spread function with a dispersion parameter
$\sigma$ equal to one sampling pixel, and should therefore be
prepared as such to the extent possible.

\subsection{Header structure and requirements}

The following primary header keywords are \textbf{required} to
be present and defined as specified, in addition
to header keywords required by the FITS standard itself:

\noindent \texttt{CRPIX1}: Shall be set to the value 1, referencing the
first sample point (``pixel'') along the wavelength axis (one-based indexing).

\noindent \texttt{CRVAL1}: Shall specify the
\textbf{base-10 logarithm} of the central wavelength
\textbf{in vacuum Angstroms} of first pixel along the wavelength axis.

\noindent \texttt{CDELT1}: Shall specify the pixel-to-pixel
increment in \textbf{base-10 logarithm} of vacuum wavelength from
one pixel to the next along the wavelength axis. \\


The following primary header keywords is \textbf{required} in the
case that the physical normalization of the templates is meaningful:

\noindent \texttt{BUNIT}: String giving the units of the template spectra. \\

The following primary header keywords are supported as
optional but \textbf{recommended}
in some combination as appropriate:

\noindent \texttt{CNAMEn}: For $n > 1$, a string giving the
name of the physical parameter coordinate along the $n^{\mathrm{th}}$ axis.

\noindent \texttt{CUNITn}: For $n > 1$, a string giving the
units of the physical parameter coordinate along the $n^{\mathrm{th}}$ axis.

\noindent \texttt{CRPIXn, CRVALn, CDELTn}: For $n > 1$, specifying
the physical parameter baselines for axes corresponding to
regularly gridded numerical physical parameters.

\noindent \texttt{PVn\_j}: For $n > 1$ and for the case of irregularly gridded
numerical physical parameters along axis $n$ , specifies the parameter value
of the $j^{\mathrm{th}}$ point along the $n^{\mathrm{th}}$ axis.
The index $j$ shall begin with 1 and increase in integer steps up to the
size of the $n^{\mathrm{th}}$ axis, with one keyword per grid step.
Note that the FITS standard currently limits the maximum size of axes
that can be represented in this manner to 99, and that leading zero-padding
is not allowed.

\noindent \texttt{PSn\_j}: For $n > 1$ and for the case of non-numerical
physical parameters along axis $n$ , specifies the parameter string value
of the $j^{\mathrm{th}}$ point along the $n^{\mathrm{th}}$ axis.
The index $j$ shall begin with 1 and increase in integer steps up to the
size of the $n^{\mathrm{th}}$ axis, with one keyword per grid step.
Note that the FITS standard currently limits the maximum size of axes
that can be represented in this manner to 99, and that leading zero-padding
is not allowed.

\noindent \texttt{Nn\_j}: For $n > 1$ and for the case of arbitrary labels
along axis $n$, specifies the label string for the $j^{\mathrm{th}}$
point along the $n^{\mathrm{th}}$ axis.
The index $j$ shall begin with 1 and increase in integer steps up to the
size of the $n^{\mathrm{th}}$ axis, with one keyword per grid step.
These keywords do not belong to the FITS standard,
and can accommodate dimensionality up to 999.

For any given axis beyond the first (wavelength) axis, the
\texttt{redmonster} code will use information from the above keywords
to construct parameter-grid baselines.  The precedence
for establishing the baseline for each axis is
first for \texttt{CRPIXn, CRVALn, CDELTn}, then for
\texttt{PVn\_j}, then for \texttt{PSn\_j}, then
for \texttt{Nn\_j}, then for a one-based integer baseline
in the absence of a valid set of keywords of any of the preceding
specified types.

\section{Implementation}

A reader routine for \texttt{ndArch} files conforming
to this proposed standard is implemented in the \texttt{redmonster}
package as \texttt{read\_ndArch}.  This routine currently resides
within a module of the same name (\texttt{read\_ndArch.py}),
however that location is
subject to change with further development and package
reorganization.  Detailed documentation of this file-reading
routine can be found in its embedded docstring.

A script called \texttt{test\_ndArch.py} is also provided,
which generates a file named \texttt{ndArch-TEST-v00.fits},
which in turn provides a unit test of the functioning of
the \texttt{read\_ndArch} routine.

\end{document}
