\documentclass[12pt]{article}
\parindent=0pt
\parskip=8pt
\begin{document}

\title{Documentation of \texttt{redmonster} Output File}

\author{Timothy A. Hutchinson}

\maketitle

\section{Introduction}

This document describes the format of the \textit{redmonster} file,
which contains the spectroscopic redshifts and classifications from
the \texttt{redmonster} software.

\section{File Format}

\subsection{File naming convention}

Files from \texttt{redmonster} will generally follow the naming scheme

\hspace*{36pt}\textit{redmonster-pppp-mmmmm.fits}

where \textit{pppp} is the 4-digit SDSS plate number, and \textit{mmmmm}
is the 5-digit MJD.  These correspond with the \textit{spPlate} input file and
and cannot be changed by the user.

In the case where a file with the above naming convention exists in the given path, 
the default behavior of the software is to overwrite it.  However, the user may 
elect to leave the older file intact, in which case the new file will be written as

\hspace*{36pt}\textit{redmonster-pppp-mmmmm-YYYY-nn-dd\_HH:MM:SS.fits}

where \textit{YYYY}, \textit{nn}, \textit{dd}, \textit{HH}, \textit{MM}, and \textit{SS}
are the year, month, day, hour, minute, and second, respectively, of the time at which
the file was written.

\subsection{File type}

All \texttt{redmonster} outputs are uncompressed FITS
files with all relevant information in the primary HDU header and the first and second
BIN tables.  The file size is approximately 20 Mb.

\subsection{File contents}

The general structure of the file is as follows:

\begin{center}
	\begin{tabular}{ | l | l | l |}
	\hline
	HDU0 & NULL & Empty \\ \hline
	HDU1 & Binary FITS Table & Object redshifts and classifications \\ \hline
	HDU2 & \texttt{nfibers} $\times$ \texttt{npix} float image & Best-fit template model for each object \\
	\hline
	\end{tabular}
\end{center}

\subsection{Header structure}

The primary HDU header is identical to that of the \textit{spPlate} files.
The header keywords are as follows:

\begin{center}
	\begin{tabular}{ | l | l | }
	\hline
	SIMPLE & FITS STANDARD \\ \hline
	BITPIX & PIXEL \\ \hline
	NAXIS & NUMBER OF AXES \\ \hline
	EXTEND & \\ \hline
	TAI & 1st row - Number of seconds since 17 Nov 1858 \\ \hline
	RA & 1st row - Right ascension of telescope boresight \\ \hline
	DEC & 1st row - Declination of telescope boresight \\ \hline
	EQUINOX & \\ \hline
	RADECSYS & \\ \hline
	AZ & 1st row - Azimuth of telescope \\
	\hline
	\end{tabular}
\end{center}

\subsection{Binary tables}

The first binary FITS table contains all redshift and classification information.
The data included is as follows, as specified by the binary table's header:

\begin{center}
	\begin{tabular}{ | l | l | }
	\hline
	Z1 & Best redshift (least $\chi_r^2$) \\ \hline
	Z2 & Second best redshift \\ \hline
	Z\_ERR1 & 1-$\sigma$ error associated with Z1 \\ \hline
	Z\_ERR2 & 1-$\sigma$ error associated with Z2 \\ \hline
	CLASS & Object type classification \\ \hline
	SUBCLASS & Best-fit template parameters \\ \hline
	FIBERED & \textit{spPlate} fiber numbers used (0-based) \\ \hline
	MINVECTOR & Coordinates of best-fit template in \textit{ndArch} file \\ \hline
	ZWARNING & Warning flags (documented elsewhere) \\ \hline
	DOF & Degrees of freedom used in calculating $\chi_r^2$ \\ \hline
	NPOLY & Number of additive polynomials used \\ \hline
	FNAME & Full name of \textit{ndArch} file of best-fit template \\
	\hline
	\end{tabular}
\end{center}

In some cases (failure of fit, small $\Delta\chi^2$, etc.), redshift values have
been set to -1.  In these cases, the associated errors have also been set to -1.

The second binary FITS table contains an \texttt{nfibers} $\times$ \texttt{npix} float
image comprised of the best-fit models (template plus polynomials at global
minimum of $\chi^2$ surface) for each fiber.  

Details for this part still need to be filled in once \texttt{io.py} is edited to write
the models as well.

\end{document}
